\documentclass[a4paper,12pt]{article}

\begin{document}

\title{Desiderata for a new EMdF and a new MQL}
\author{Ulrik Sandborg-Petersen}
\date{2019/05/29}

\section{Introduction}

This paper describes, motives, explains, and exemplifies some
desiderata for a new MQL and EMdF model.

\section{New EMdF}

\subsection{Introduction}

The following are proposed as additions to the high-level abstraction
of EMdF:

\begin{enumerate}
\item \textbf{Virtual objects:} That is, objects which have a place in
  the monad stream, because they are anchored in the text, but which
  are not part of the main text at that point in the monad stream.
  
\item \textbf{Extra-textual objects:} That is, objects which are not
  part of the main text, but which also do not have a natural
  anchoring at a specific location in the text.  Rather, they are
  outside the text, and are referred to by objects in the text, and
  perhaps also the other way around.

\item \textbf{Better declaration of properties of and relationships
  between monad set features:} It should be possible to declare
  explicitly certain properties of monad set features, as well as
  declaring that certain monad set features are to be interpreted in
  the same context.

\item \textbf{Better typing of feature values:} Certain types of
  feature values should be better handled.  See below for concrete
  suggestions.

\item \textbf{Better handling of UPDATE OBJECT:} It should be possible
  to update the monad set of an object.
  
\end{enumerate}

\noindent In order to implement these high-level abstractions, the
following are proposed as additions to the concrete implementation of
EMdF:

\begin{enumerate}
\item \textbf{Named monad streams:} Right now, there is only one monad
  stream.  It is currently unnamed, but I propose calling it ``main''
  in the future.  This should always be declared implicitly at
  database creation time.

  We should add the ability to declare separate monad streams and call
  them something else.

  It should also be possible to declare at object type creation time
  that a given set-of-monads feature belongs to a given, previously
  explicitly or implicitly declared, monad stream.

  It should also be possible to declare that a monad set feature
  belongs to ``\texttt{ANY}'' monad stream.  That is, it should be
  possible not to assign a specific monad stream to a given monad set
  feature.  This should be the default.

  This will gain us these advantages:
  \begin{enumerate}
  \item A monad set feature can be declared to belong to a certain
    monad stream.  This will make the relationships explicit.  This
    explicitation is useful in two ways:
    \begin{enumerate}
      \item It helps the database designer think about his textual
        domain.
      \item It helps the EMdF and MQL layers detect errors in usage.
      \item It helps the MQL layer, and possibly also the EMdF layer,
        to use objects correctly and more efficiently.
    \end{enumerate}

  \item Separation of different textual streams:

    \begin{enumerate}
    \item Sometimes, different kinds of text in a document all belong
      to the document, but are of entirely different natures, and
      actually are separate parts of the same document.  Examples
      include: Footnotes, Apparatuses, Inserts, and other parts of the
      document which have an anchoring in the main text, but which are
      not part of the main text.
    \item Sometimes, a database needs to contain extra-textual data
      which is not part of the main text, nor has a natural placement
      in the document reading order, but which is nevertheless
      important information for the reading / interpretation of the
      document.  Having the ability to declare separate monad streams
      for these kinds of objects is a real advantage.
    \end{enumerate}
  \end{enumerate}

\item \textbf{Declarared monad set range types:} It should be
  possible, at object type creation time, to declare that a given
  monad set feature (not just ``monads'', but all features of type SET
  OF MONADS) is of a given range type: SINGLE MONAD, SINGLE RANGE, or
  MULTIPLE RANGE set of monads.
  
\item \textbf{Observed monad set range types:} The EMdF engine should
  keep track of the range types of all monad sets of all objects in
  all object types, both the ``monads'' feature and all other monad
  set features.  That is, when an object is created, the EMdF engine
  should store the largest range type that is used by the actual
  objects for this feature.  This should be according to this cline:

  \begin{center}
    \texttt{MULTIPLE RANGE} $>$ \texttt{SINGLE RANGE} $>$ \texttt{SINGLE MONAD}
  \end{center}
  
  \noindent This will help the MQL engine decide optimizations better.
  
\item \textbf{Virtual object metadata:} At object creation time (not
  object \textit{type} creation time), it should be possible to
  declare that a given object of the given object type is ``virtual'',
  in the sense that, while it has a placement in the monad stream, it
  is not necessarily a part of the main text of that monad stream.
  Instead, it sits ``between'' two monads.  Thus, if the two monads
  between which it sits are adjacent, the virtual object has zero
  extent.  If the two monads between which it sits are not adjacent,
  then the virtual object \textit{is} part\_of the monad stream by at
  least one monad.

  It should also be possible to have more than one virtual object
  sitting in between two monads, yet still have an ordering on those
  objects.
 
  A possible way to implement this is outlined below.
 
\item \textbf{Virtual object adjacency rules:} Given that a virtual
  object has a placement in the monad stream, but may have zero
  extent, it becomes necessary to define adjacency rules between
  ``normal'' objects and virtual objects.

  A possible way to implement this is also outlined below.

\item \textbf{Better handling of feature types:} Certain
  possibilities have suggested themselves over the course of using
  the current typing schema.  These include:
  
  \begin{enumerate}
    
  \item \textbf{Arbitrary strings as enumeration constants:} It
    should be possible to define enumeration constants by means of
    arbitrary strings, not just C identifiers.
    
  \item \textbf{Arbitrary strings in lists of enumeration constants:}
    Of course, when a feature type is ``LIST OF
    \textit{enumeration}'', then it should also be possible to use
    strings in those lists.
      
  \item \textbf{LIST OF STRING:} It should be possible to declare that
    a given feature is of type LIST OF STRING, meaning a list of
    arbitrary strings.
      
  \end{enumerate}
  
\end{enumerate}

\subsubsection{Implementing monad streams}

\paragraph{Metadata}

A new metadata table should be created:

\bigskip

  \begin{center}
\noindent\begin{tabular}{|l|l|}
\hline
\textbf{type\_id : id\_d\_t} & \textbf{MonadStreamName : STRING}\\
\hline
\end{tabular}
  \end{center}

\bigskip

\noindent This should be created at database creation time.  At the
same time, entries designating the ``main'' and ``any'' monad streams
should be created.  The ``any'' monad stream should have the special
type id\_d\_t ``\texttt{NIL}'' assigned.

\begin{table}[h]
  \begin{center}
  \caption{Initial data for the ``monad\_streams'' table.}
\medskip  
\noindent\begin{tabular}{|l|l|}
\hline
\textbf{type\_id : id\_d\_t} & \textbf{MonadStreamName : STRING}\\
\hline
0  & any \\
\hline
10001 & main \\
\hline
\end{tabular}
  \end{center}
\end{table}


The strings should be restricted to being C identifiers by the MQL
layer.  The strings should be case-IN-sensitive.

\paragraph{MQL enhancements}

New MQL statements should be made for CREATE and DROP operations on
monad streams.  Also, it should be possible to declare that a given
monad set feature is in a particular monad stream.  See Section
\ref{sec:MonadStreams} below.

\subsubsection{Implementing Virtual objects}

Virtual objects are always of a particular object type, which may also
contain non-virtual or ``normal'' objects.

When creating the virtual object, its ``monads'' feature should be
given a monad set.  Because the object is marked as ``virtual'', this
monad set will be interpreted specially by the MQL engine.

In effect, the monad set should be interpreted such that:

\begin{enumerate}
  \item \textbf{Monad set width:} The interpreted width of the monad
    should be one less than the actual width.  That is, the width of a
    monad set $m$ is given by the formula:

    \[m.\mathrm{last} - m.\mathrm{first}\]
    
    In the usual case, the width is, of course, given by the formula:
    
    \[m.\mathrm{last} - m.\mathrm{first} + 1\]

    \item \textbf{Monad placement:} The monad set should be
      interpreted such that it's real \textit{first} monad should be
      interpreted to be placed just \textit{after} the real first
      monad.  One can think of it in terms of the interpreted first
      monad being on (virtual) monad $m.\mathrm{first} + 0.5$, which
      is between $m.\mathrm{first}$ and $m.\mathrm{first} + 1$
      
      Similarly, the real \textit{last} monad should be interpreted as
      being just \textit{before} the last monad + 1, but still after
      the first monad.  One can think of it as being at (virtual)
      monad $m.\mathrm{last} + 0.5$

      Thus if $m.\mathrm{first} = m.\mathrm{last}$, the monad set will
      have zero width, and will fall between the (real) monads
      $m.\mathrm{first}$ and $m.\mathrm{last} + 1$.

      If, on the other hand, $m.\mathrm{first} < m.\mathrm{last}$,
      then the monad set will start after $m.\mathrm{first}$ (namely
      at ``virtual monad'' $m.\mathrm{first} + 0.5$, and will also end
      before $m.\mathrm{last} + 1$, namely on $m.\mathrm{last} + 0.5$
      In this casem since $m.\mathrm{first} < m.\mathrm{last}$, it
      will be the case that the monad set has at least one monad which
      sits in the real monad stream.

      In either case, MQL should be made smart enough to know that the
      virtual monad set is part\_of the surrounding Substrate, and
      also that its adjacency is \textit{after} $m.\mathrm{first}$ and
      \textit{before} $m.\mathrm{last} + 1$.

      \item \textbf{Ordering of more than one virtual object starting
        on the samme virtual monad:} It should be possible to declare
        in both EMdF and MQL how multiple virtual objects starting on
        the same virtual monad should be ordered, both as regards
        adjacency and as regards containment.

        This is necessary because the same virtual monad ``slot'' may
        need to contain more than one virtual object, and we want to
        be able to declare that such and such a virtual object is
        before, after, or contained inside another virtual object
        sitting on the same virtual monad.

        This can be done in a number of ways.  Here it is suggested to
        use another monad set feature, and declare it to be the
        arbitrating monad set in terms of ordering, or its ``secondary
        monad set''.

        That is, it should be possible at object type creation time to
        add another monad set feature $M$, other than ``monads''
        (this part is already possible), and then declare that this
        monad set feature $M$ is a ``secondary'' monad set,
        arbitrating order and containment in the case of more than one
        virtual object sitting on the same monad (this is the novel
        part).

        In fact, this is useful not just for virtual objects, but also
        for extra-textual objects.

        The rules of adjacency and containment are outlined below.
      
\end{enumerate}

\subsubsection{Virtual object rules for adjacency and containment}

As noted above, it should be possible, at object type creation time,
to declare any monad set feature other than ``monads'' to be the
``secondary'' monad set of the object.  This should make it possible
for MQL to arbitrate between objects (virtual or normal) that start on
the same monad (virtual or real) as to adjacency in case there is a
tie on the primary monad set.

When running an MQL query, it is already possible to declare that a
given object\_block should use such and such a monad set feature for
checking part\_of, starts\_in, and overlap with the substrate.  This
monad set also currently becomes the new monad set used for checking
the adjacency relations.

The syntax should be extended such that a given monad set feature should
be able to be declared to be used for:

\begin{enumerate}
\item \textbf{Adjacency:} Be used for adjacency (before and after).
\item \textbf{Containment:} Be used for containment (inside, namely
  part\_of, starts\_in, and overlap of the inner \texttt{blocks}).
\end{enumerate}

\noindent Moreover, for adjacency, a secondary monad set feature
should be able to be picked, in case the primrary monad set feature
makes a tie.

Notice that the primary and secondary monad sets should be able to be
picked in the topographic query, hence it is not strictly necessary to
declare a certain monad set feature to be the secondary feature at
object type creation time.  However, the significance of actually
declaring a certain monad set feature to be the secondary monad set
feature is that it makes it possible for MQL to automatically pick the
right one in case it needs it.

MQL aside, the rules should be as follows:

An object (virtual or not) should always be compared on its primary
monad set first.  Then, if this gives a tie, we should check if there
is a secondary monad set.  If not, then the objects are neither before
nor after each other.  If, on the other hand, there is a secondary
monad set, then this secondary monad set should be tried.

Notice that the before/after relations have three possible outcomes:
before, after, and neither.  Notice also that for the containment
relations (part\_of, starts\_in, and overlaps), there is no middle
ground: It is always a ``yes/no'' or ``true/false'' answer.

Therefore, the rule of using the secondary monad set in case of a tie
only applies to the before/after relation, not to the containment
relations.

\subsubsection{Notes on implementing virtual objects and extra-textual objects}

\begin{itemize}
\item When comparing two ``normal'' objects, the rules of adjacency
  and containment above should be used.

\item When comparing two ``virtual'' objects, the rules of adjacency
  and containment above should also be used directly on the monad sets
  involved.

\item when comparing a ``normal'' object $N$ with a ``virtual'' object
  $V$, the following rules apply:

  \begin{enumerate}
    \item If $V.\mathrm{first} \geq N.\mathrm{last}$, then $V$ is after $N$.
    \item If $V.\mathrm{last} < N.\mathrm{first}$, then $V$ is before $N$.
    \item If $V.\mathrm{first} \leq N.\mathrm{first} \wedge V.\mathrm{last} \geq N.\mathrm{last}$, then $V$ is after $N$.
    \item If $V.\mathrm{first} \leq N.\mathrm{first} \wedge V.\mathrm{last} \leq N.\mathrm{last}$, then $V$ is before $N$.
    \item \ldots{} @@@ TO BE COMPLETED....
  \end{enumerate}

\end{itemize}

\subsubsection{Implementing feature enhancements}

\section{New MQL}

\subsection{Introduction}

These items are discussed below:

\begin{enumerate}
\item Monad streams
\item Monad set range type
\item Strings as enumeration constants
\item New feature type: LIST OF STRING
\item New statement: HARVEST
\item Enhancements to UPDATE OBJECT
\end{enumerate}

\subsection{Monad streams}\label{sec:MonadStreams}

\subsubsection{Introduction}

\subsubsection{CREATE MONAD STREAM}

Suggested syntax example:

\begin{verbatim}
CREATE MONAD STREAM Footnotes
GO
\end{verbatim}

\noindent so, basically, the three tokens ``CREATE'', ``MONAD'', and
``STREAM'', followed by an IDENTIFIER.

Note that \texttt{STREAM} is not currently a keyword, so the list of
keywords will have to be expanded.

\subsubsection{DROP MONAD STREAM}

It should be possible to drop a monad stream, provided no feature
currently uses this monad stream.  If a feature on some object type is
found to use the current monad stream, then an error should be thrown,
and the DROP MONAD STREAM statement should not succeed.

Suggested syntax example:

\begin{verbatim}
DROP MONAD STREAM Footnotes
GO
\end{verbatim}

\subsubsection{Enhancements to CREATE OBJECT TYPE}

Both set ``monads'' feature and all other ``set of monads'' features should be able to be declared to be in a specific monad stream.

Suggested syntax example:

\begin{verbatim}
CREATE OBJECT TYPE
WITH monads IN MONAD STREAM Main
[Footnote
   // This anticipates the HAVING <range-type> SETS syntax
   footnote_text_som : SET OF MONADS 
                       HAVING SINGLE RANGE SETS 
                       IN MONAD STREAM Footnotes
]
GO
\end{verbatim}

\subsection{Monad set range type}

\subsubsection{Introduction}

\subsection{Enhancements to CREATE OBJECT TYPE}

\subsection{Strings as enumeration constants}

\subsubsection{CREATE ENUMERATION}
\subsubsection{UPDATE ENUMERATION}
\subsubsection{Enumeration usages}

\subsection{New feature type LIST OF STRING}

\subsubsection{CREATE OBJECT TYPE}

A new type should be created, both in the MQL grammar and everywhere
else it is necessary, e.g., in emdf.h, infos.{cpp,h},
emdf\_value.{cpp,h}, and elsewhere.

This new type should be added to the CREATE OBJECT TYPE grammar and
Statement implementation, as well as to the EMdF layer.

\subsubsection{UPDATE OBJECT TYPE}

It should be possible to ADD a feature with type type LIST OF STRING
to an existing object type.

\subsubsection{CREATE OBJECT}

When creating a normal or virtual object, it should be possible to set
a given feature to be a list of string.  The curren syntax for lists
should be re-used, i.e., parenthesis-surrounded, commma-separated
list.

\subsubsection{Feature usages}

In the topographic part of MQL, it should be possible to search
features of type \texttt{LIST OF STRING} with ``\texttt{IN}
\textit{list-of-string}'' and ``\texttt{HAS} \textit{string}''.

This insolves both feature comparisons and object reference usages, as
well as EMdFValue.

\subsection{New HARVEST statement}

\subsection{Enhancements to UPDATE OBJECT}

\subsubsection{Updating the ``monads'' feature with UPDATE OBJECT}

The UPDATE OBJECT statement should be changed so that it is also able
to update the ``monads'' feature, i.e., the privileged monad set.

The details of the syntax should be described in the MQL documentation
before implementation.

\subsubsection{Combining HARVEST and UPDATE OBJECTS}

\end{document}
